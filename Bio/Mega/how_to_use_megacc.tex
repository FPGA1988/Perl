% Created 2014-03-26 Wed 23:16
\documentclass[11pt]{ctexart}
\usepackage[utf8]{inputenc}
\usepackage[T1]{fontenc}
\usepackage{fixltx2e}
\usepackage{graphicx}
\usepackage{longtable}
\usepackage{float}
\usepackage{wrapfig}
\usepackage{soul}
\usepackage{textcomp}
\usepackage{marvosym}
\usepackage{wasysym}
\usepackage{latexsym}
\usepackage{amssymb}
\usepackage{hyperref}
\tolerance=1000
\providecommand{\alert}[1]{\textbf{#1}}

\title{如何使用Mega cc}
\author{GRC(扬眉剑)}
\date{\today}
\hypersetup{
  pdfkeywords={},
  pdfsubject={},
  pdfcreator={Emacs Org-mode version 7.9.3f}}

\begin{document}

\maketitle

\setcounter{tocdepth}{3}
\tableofcontents
\vspace*{1cm}

\section{github地址:}
\label{sec-1}

发现任何错误,或者不当的地方,请联系我。
\href{https://github.com/gaorongchao/Perl/tree/master/Bio/Mega}{https://github.com/gaorongchao/Perl/tree/master/Bio/Mega}
\section{下载}
\label{sec-2}

\href{http://www.megasoftware.net/megaccusage.php}{http://www.megasoftware.net/megaccusage.php}

先从上面的网址下载,阅读上面的下载协议。
然后选“Accept Agreement”,然后就开始下载了。
下载以后我们得到“M6CC.zip”的文件。
然后解压。

解压以后得到几个文件。

\begin{verbatim}
M6CC.exe
M6Proto.exe #这两个是主程序,是我们要用到的

MEGA-CC-Quick-Start-Tutorial.pdf # 这个是使用的教程
Usage Agreement.pdf # 使用协议,不用看

Examples # 这个文件夹提供了我们学习MEGACC所需要的文件
\end{verbatim}
\section{使用}
\label{sec-3}

使用过程是从解压以后得到的PDF手册翻译加工整理而来。
\subsection{安装}
\label{sec-3-1}

你可以在上面得到的解压文件中直接使用。
也可以直接把起作用的两个执行文件M6CC.exe 和 M6Proto.exe
拷贝到你喜欢的文件夹中。
下面的两个实例需要你也把Example文件也拷贝过去。
\subsection{输入文件}
\label{sec-3-2}


\begin{verbatim}
1:分析配置文件:也就是你要用MEGA的什么参数进行分析的一个参数设定的集合。
    这个是用MEGA-Proto来生成的。
    生成的文件是一个 .mao为后缀的文件。
2:数据文件(下面的任意一个都可以)
    * Multiple sequence alignment in MEGA or Fasta format. 
    * Distance matrix in MEGA format. 
    * Unaligned sequences in Fasta format (f or alignment only)
3:树文件(某些分析需要) .nwk文件格式。
\end{verbatim}
\subsection{输出文件}
\label{sec-3-3}

一般情况下生成两种输出文件


\begin{verbatim}
1. Calculation-specific results file (Newick file, distance 
ORG-LIST-END-MARKER
matrix,…). 
2. A summary file with additional info (likelihood, SBL,…). 
ORG-LIST-END-MARKER
* Some analyses produce additional output (bootstrap consensus 
tree).
\end{verbatim}
输出的文件夹和文件名称


\begin{verbatim}
1:默认和输入文件在同一个文件夹
2:如果要改变文件夹或者文件名称,那么用-o 选项。
3: If no output filename is specified, MEGA-CC will assign a unique 
name.
\end{verbatim}
错误和警告:

如果MEGA-CC产生了错误或者警告信息。那么会出现在summary file文件中。
\subsection{运行MEGA-CC}
\label{sec-3-4}

   \texttt{DEADLINE:} \textit{2014-03-26 Wed}

用命令行运行非常简单。

\begin{verbatim}
M6CC.exe -a options.mao -d alignment.meg -o outFile
\end{verbatim}
也可以用一些脚本程序来运行,比如Perl,Python

\begin{verbatim}
exec('M6CC.exe -a options.mao -d alignment.meg -o outFile');
\end{verbatim}
MEGA-CC附带完善的文件迭代系统,来处理多个文件,而不用脚本来帮忙。
具体可以参考下面第二个实例。

其他的应用也可以调用MEGA-CC:

\begin{verbatim}
status = CreateProcess("M6CC.exe...");
\end{verbatim}
如果要查看更多的命令选项,那么从命令行中调用M6CC.exe -h
\subsection{MEGA-Proto (分析模版)}
\label{sec-3-5}


\begin{verbatim}
第一:和图形界面版本一样的外观
第二:生成MEGA分析所需要的选项文件
第三:没有计算能力,只是一个模拟的过程
\end{verbatim}
双击打开MEGA-Proto以后,图形界面会引导你进行一下步骤。

第一步:选择输入文件类型。
包含四种类型。

\begin{verbatim}
Nucleotide (non-coding)
Nucleotide (coding)
Protein (amino-acid)
Distance matrix (MEGA format)
\end{verbatim}

第二步:从上面的菜单中选择一项分析过程。

第三步:调整分析的参数设置。

第四步:保存你设置好的MEGA分析的选项到一个文件。
\subsection{Demo1:实例1}
\label{sec-3-6}

本实例展现了如何用MEGA-Proto 和MEGA-CC来完成
“Maximum Likelihood phylogeny reconstruction”。

第0步:准备文件

为了能够正确的完成这个实例,你需要确认你有了M6CC.exe,
以及M6Proto.exe这两个程序(我们上面下载的就是)。
事例所需要的文件就是我们上面解压得到的。

\begin{verbatim}
第1步:双击或者右键打开MEGA-Proto.exe。
第2步:选择输入的数据类型。这里我们用的是默认设置。Nucleotide (non-coding)。
第3步:在菜单栏中选择:Phylogeny => Construct/Test Maximum Likelihood Tree
第4步:调整参数,然后点击“Save Settings”。把文件“mlDemo.mao”保存在当前文件夹。
第5步:打开一个命令行界面。也就是win+R,cmd。然后用cd命令切换到M6CC.exe所在的文件夹。
第6步:用M6CC.exe执行程序来分析文件。
      M6CC.exe -a mlDemo.mao -d Examples\Crab_rRNA.meg -o demoResults
第7步:程序开始执行。会有进度的显示。执行完成以后退出。
最后 :分析得到3个输出文件。
    * demoResult.nwk
      这个文件是用我们给定的设置参数得到的Maximum Likelihood 树。
    * demoResult_consensus.nwk
      这个文件是Mega 从所有的bootstrap sample trees中得到的bootstrap consensus树。
    * demoResult_summary.txt
      这个文件给出了分析数据:比如log likelihood value of the Maximum Likelihood tree,ts/tv ratio etc...
\end{verbatim}
\subsection{Demo2: 实例2}
\label{sec-3-7}

下面这个例子展示了,如何用MEGA-CC中的文件迭代系统
来用一个配置好的文件来处理多个输入文件。

第0步:启动
这个就是第一个Demo1中的前5个过程,如果还搞不定的话,自己去复习。


\begin{verbatim}
第1步:新建一个文件命名为 demo2Data.txt 这里面包含我们要处理的多个文件。
      一行是一个文件。
      在这个文件中,对两个文件Grab_rRNA.meg 和Drosophila_Adh.meg两个实例
      文件添加完全的路径。
第2步:然后在命令行中用如下命令调用MEGA-CC:
      M6CC.exe -a mlDemo.mao -d demo2Data.txt
第3步:然后分析会启动。一个个的处理文件。处理的进度会显示在命令行界面中。
最后 :分析程序会对每一个输入文件产生一个输出结果。
      在这个例子中,相同的分析配置文件用在每一个文件中。
\end{verbatim}
\subsection{自我实例}
\label{sec-3-8}

下面是一个用Perl调用的小脚本:

\begin{verbatim}
use strict;
use warnings;
use utf8;
use 5.16.3;


my @files = glob "*.fasta";
foreach my $file (@files)
{
        exec("M6CC.exe -a huashu.mao -d $file -o $file.out");
}
\end{verbatim}
使用的huashu.mao都在github上。

\end{document}

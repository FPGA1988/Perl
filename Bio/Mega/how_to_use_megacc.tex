% Created 2014-03-29 Sat 23:19
\documentclass{ctexart}
\usepackage[utf8]{inputenc}
\usepackage[T1]{fontenc}
\usepackage{fixltx2e}
\usepackage{graphicx}
\usepackage{longtable}
\usepackage{float}
\usepackage{wrapfig}
\usepackage{soul}
\usepackage{textcomp}
\usepackage{marvosym}
\usepackage{wasysym}
\usepackage{latexsym}
\usepackage{amssymb}
\usepackage{hyperref}
\tolerance=1000
\usepackage{listings}
\providecommand{\alert}[1]{\textbf{#1}}

\title{如何使用Mega cc}
\author{GRC(扬眉剑)}
\date{\today}
\hypersetup{
  pdfkeywords={},
  pdfsubject={},
  pdfcreator={Emacs Org-mode version 7.9.3f}}

\begin{document}

\maketitle

\setcounter{tocdepth}{3}
\tableofcontents
\vspace*{1cm}
\section{github地址:}
\label{sec-1}

\href{https://github.com/gaorongchao/Perl/tree/master/Bio/Mega}{https://github.com/gaorongchao/Perl/tree/master/Bio/Mega}

发现任何错误,或者不当的地方,请联系我.
\section{下载}
\label{sec-2}

\href{http://www.megasoftware.net/megaccusage.php}{http://www.megasoftware.net/megaccusage.php}

先从上面的网址下载,阅读上面的下载协议。
然后选“Accept Agreement”,然后就开始下载了。
下载以后我们得到“M6CC.zip”的文件。
然后解压。

解压以后得到几个文件。
\begin{itemize}
\item M6CC.exe
\item M6Proto.exe \#这两个是主程序,是我们要用到的
\item MEGA-CC-Quick-Start-Tutorial.pdf \# 这个是使用的教程
\item Usage Agreement.pdf \# 使用协议,不用看
\item Examples \# 这个文件夹提供了我们学习MEGACC所需要的文件
\end{itemize}
\section{使用}
\label{sec-3}

使用过程是从解压以后得到的PDF手册翻译加工整理而来。
\subsection{安装}
\label{sec-3-1}

你可以在上面得到的解压文件中直接使用。
也可以直接把起作用的两个执行文件M6CC.exe 和 M6Proto.exe
拷贝到你喜欢的文件夹中。
下面的两个实例需要你也把Example文件也拷贝过去。
\subsection{输入文件}
\label{sec-3-2}

1:分析配置文件:也就是你要用MEGA的什么参数进行分析的一个参数设定的集合。
这个是用MEGA-Proto来生成的。
生成的文件是一个 .mao为后缀的文件。

2:数据文件(下面的任意一个都可以)
\begin{itemize}
\item Multiple sequence alignment in MEGA or Fasta format.
\item Distance matrix in MEGA format.
\item Unaligned sequences in Fasta format (f or alignment only)
\end{itemize}

3:树文件(某些分析需要) .nwk文件格式。
\subsection{输出文件}
\label{sec-3-3}

一般情况下生成两种输出文件

\begin{itemize}
\item 1. Calculation-specific results file (Newick file, distance
\end{itemize}
  matrix,…). 
\begin{itemize}
\item 2. A summary file with additional info (likelihood, SBL,…).
\end{itemize}
  , Some analyses produce additional output (bootstrap consensus 
  tree).
 
输出的文件夹和文件名称

\begin{itemize}
\item 1:默认和输入文件在同一个文件夹
\item 2:如果要改变文件夹或者文件名称,那么用-o 选项。
\item 3: If no output filename is specified, MEGA-CC will assign a unique 
  name.
\end{itemize}

错误和警告:

如果MEGA-CC产生了错误或者警告信息。那么会出现在summary file文件中。
\subsection{运行MEGA-CC}
\label{sec-3-4}

用命令行运行非常简单。

\lstset{frame=trBL,frameround=fttt,breaklines=true,language=Perl}
\begin{lstlisting}
M6CC.exe -a options.mao -d alignment.meg -o outFile
\end{lstlisting}
也可以用一些脚本程序来运行,比如Perl,Python。
这里我们用exec发现会出问题,所以这里改成用system来进行调用。

\lstset{frame=trBL,frameround=fttt,breaklines=true,language=Perl}
\begin{lstlisting}
#exec('M6CC.exe -a options.mao -d alignment.meg -o outFile'); 
system ('M6CC.exe -a options.mao -d alignment.meg -o outFile');
\end{lstlisting}
MEGA-CC附带完善的文件迭代系统,来处理多个文件,而不用脚本来帮忙。
具体可以参考下面第二个实例。

其他的应用也可以调用MEGA-CC:

\lstset{frame=trBL,frameround=fttt,breaklines=true,language=Perl}
\begin{lstlisting}
status = CreateProcess("M6CC.exe...");
\end{lstlisting}
如果要查看更多的命令选项,那么从命令行中调用M6CC.exe -h
\subsection{MEGA-Proto (分析模版)}
\label{sec-3-5}

MEGA-Proto有一下特点:
\begin{itemize}
\item 第一:和图形界面版本一样的外观
\item 第二:生成MEGA分析所需要的选项文件
\item 第三:没有计算能力,只是一个模拟的过程
\end{itemize}
双击打开MEGA-Proto以后,图形界面会引导你进行一下步骤。

第一步:选择输入文件类型。
包含四种类型。

\lstset{frame=trBL,frameround=fttt,breaklines=true,language=Perl}
\begin{lstlisting}
Nucleotide (non-coding)
Nucleotide (coding)
Protein (amino-acid)
Distance matrix (MEGA format)
\end{lstlisting}

第二步:从上面的菜单中选择一项分析过程。

第三步:调整分析的参数设置。

第四步:保存你设置好的MEGA分析的选项到一个文件。
\subsection{Demo1:实例1}
\label{sec-3-6}

本实例展现了如何用MEGA-Proto 和MEGA-CC来完成
“Maximum Likelihood phylogeny reconstruction”。

\begin{itemize}
\item 第0步:准备文件。
\end{itemize}
为了能够正确的完成这个实例,你需要确认你有了M6CC.exe,
以及M6Proto.exe这两个程序(我们上面下载的就是)。
实例所需要的文件就是我们上面解压得到的。
\begin{itemize}
\item 第1步:双击或者右键打开MEGA-Proto.exe。
\item 第2步:选择输入的数据类型。这里我们用的是默认设置。Nucleotide (non-coding)。
\item 第3步:在菜单栏中选择:Phylogeny => Construct/Test Maximum Likelihood Tree
\item 第4步:调整参数,然后点击“Save Settings”。把文件“mlDemo.mao”保存在当前文件夹。
\item 第5步:打开一个命令行界面。也就是win+R,cmd。然后用cd命令切换到M6CC.exe所在的文件夹。
\item 第6步:用M6CC.exe执行程序来分析文件。
\end{itemize}

\lstset{frame=trBL,frameround=fttt,breaklines=true,language=Perl}
\begin{lstlisting}
M6CC.exe -a mlDemo.mao -d Examples\Crab_rRNA.meg -o demoResults
\end{lstlisting}
\begin{itemize}
\item 第7步:程序开始执行。会有进度的显示。执行完成以后退出。
\item 最后 :分析得到3个输出文件。
\end{itemize}

\begin{verbatim}
* demoResult.nwk
这个文件是用我们给定的设置参数得到的Maximum Likelihood 树。
* demoResult_consensus.nwk
这个文件是Mega 从所有的bootstrap sample trees中得到的bootstrap consensus树。
* demoResult_summary.txt
这个文件给出了分析数据:比如log likelihood value of the Maximum Likelihood tree,ts/tv ratio etc...
\end{verbatim}
\subsection{Demo2: 实例2}
\label{sec-3-7}

下面这个例子展示了,如何用MEGA-CC中的文件迭代系统,
用同一个配置文件(也就是MEGA-Proto得到的文件)来处理多个输入文件。

第0步:启动

这个就是第一个Demo1中的前5个过程,如果还搞不定的话,自己去复习。

第1步:新建一个文件命名为 demo2Data.txt 这里面包含我们要处理的多个文件。
        一行是一个文件。
        在这个文件中,有两个文件“Grab\_rRNA.meg” 和“Drosophila\_Adh.meg”。
        文件添加完全的路径。

第2步:然后在命令行中用如下命令调用MEGA-CC:

\lstset{frame=trBL,frameround=fttt,breaklines=true,language=Perl}
\begin{lstlisting}
M6CC.exe -a mlDemo.mao -d demo2Data.txt
\end{lstlisting}
        
      
       上面的命令行没有指定输出的文件夹,以及文件名称。不过不用担心。
       所有的结果都会根据你的文件名来命名,并且输出到“M6CC\_Out”文件夹中。

第3步:然后分析会启动。一个个的处理文件。处理的进度会显示在命令行界面中。

最后 :分析程序会对每一个输入文件产生一个输出结果。
        在这个例子中,相同的分析配置文件用在每一个文件中。
\subsection{自我实例}
\label{sec-3-8}

下面是一个用Perl调用的小脚本:

\lstset{frame=trBL,frameround=fttt,breaklines=true,language=Perl}
\begin{lstlisting}
use strict;
use warnings;
use utf8;
use 5.16.3;

my @files = glob "*.fasta";
foreach my $file (@files)
{
        exec("M6CC.exe -a huashu.mao -d $file -o $file.out");
}
\end{lstlisting}
使用的huashu.mao都在github上。
但是如果直接用上面的Perl程序来画树,画完一个树以后就停止了。
经过测试发现,把上面程序中的“exec”换成“system”命令以后,程序完美运行。

同时我们也可以用MEGA-cc自带的多文件处理功能。
Perl脚本用来提取所有需要处理文件的文件名到file.txt,这个文件名要包含完全的路径。

\lstset{frame=trBL,frameround=fttt,breaklines=true,language=Perl}
\begin{lstlisting}
use strict;
use warnings;
use utf8;
use 5.16.3;

my      $out_out = "file.txt";
open  my $out, '>', $out_out or die  "Fail open $out_out\n";
my @files = glob "*.fasta";
foreach my $file (@files)
{
        print $out "D:\\Less_less_region\\$file\n";
        #exec("M6CC.exe -a huashu.mao -d $file -o $file.out");
}
close  $out;
\end{lstlisting}
然后用上面的多个文件处理的方法:

\lstset{frame=trBL,frameround=fttt,breaklines=true,language=Perl}
\begin{lstlisting}
M6CC.exe -a huashu.mao -d file.txt
\end{lstlisting}
但是用上面的方法来处理文件的时候,只有一半的文件被处理,基本上是处理一个,跳过一个。
不知道是什么原因。我能想到的解决办法是:在上面的输出文件名称到file.txt的时候,
每一个文件都输出两遍。仅仅是一种解决方案,没有找到真正原因。

\lstset{frame=trBL,frameround=fttt,breaklines=true,language=Perl}
\begin{lstlisting}
use strict;
use warnings;
use utf8;
use 5.16.3;

my      $out_out = "file.txt";
open  my $out, '>', $out_out or die  "Fail open $out_out\n";
my @files = glob "*.fasta";
foreach my $file (@files)
{
        print $out "D:\\Less_less_region\\$file\n";
        print $out "D:\\Less_less_region\\$file\n";
        #exec("M6CC.exe -a huashu.mao -d $file -o $file.out");
}
close  $out;
\end{lstlisting}
\section{mao 文件简单解析}
\label{sec-4}

mao文件是我们用M6Proto.exe通过模拟分析得到的一个参数列表。
但是他的本质就是一个文本文件。我们可以用文本编辑器打开。

\lstset{frame=trBL,frameround=fttt,breaklines=true,language=Perl}
\begin{lstlisting}
; Please do not edit this file! If this file is modified, results are unpredictable.
; Instead of modifying this file, simply create a new MEGA Analysis Options file by using the MEGA Prototyper.
[ MEGAinfo ]
ver=0
[ DataSettings ]
datatype=snNucleotide
containsCodingNuc=False
missingBaseSymbol=?
identicalBaseSymbol=.
gapSymbol=-
[ ProcessTypes ]
ppInfer=true
ppNJ=true
[ AnalysisSettings ]
Analysis=Phylogeny Reconstruction
Scope=All Selected Taxa
Statistical Method=Neighbor-joining
Phylogeny Test=====================
Test of Phylogeny=Bootstrap method
No. of Bootstrap Replications=500
Substitution Model=====================
Substitutions Type=Nucleotide
Model/Method=p-distance
Substitutions to Include=d: Transitions + Transversions
Rates and Patterns=====================
Rates among Sites=Uniform rates
Gamma Parameter=Not Applicable
Pattern among Lineages=Same (Homogeneous)
Data Subset to Use=====================
Gaps/Missing Data Treatment=Pairwise deletion
Site Coverage Cutoff (%)=Not Applicable
\end{lstlisting}
虽然文件的第一行,不让我们修改,但是我们仔细看一下每一行的意思以后。
完全可以进行小的改动。这样就不必要每一个小的改动,都去使用M6Proto.exe
这个程序。

比如,上面有一行 No. of Bootstrap Replications=500。
这个我们非常容易理解,就是设置Bootstrap的次数,如果你想改成1000,那就直接从mao文件
中修改吧。
\section{在Linux下如何使用}
\label{sec-5}

官方的说法是现在只支持windows系统,暂时还不支持Mac和Linux。
Mega只能用最大4G的内存。

MEGA CC is developed for use on Microsoft Windows operating systems, 
including XP, Vista, Windows 7, and Windows 8. 
The version is limited to 32-bit execution, but should run fine on 64-bit systems.
32-bit limitations will still apply ex. 
MEGA can only use max 4gb of memory. 
At this time Mac and Linux are not supported.

\end{document}
